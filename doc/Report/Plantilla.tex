\documentclass[a4paper]{article}

\usepackage[utf8]{inputenc}
\usepackage[english]{babel}
\usepackage{graphics}
\usepackage{caption}
\usepackage{subcaption}
\usepackage[demo]{graphicx}
\usepackage{enumitem}
\usepackage{longtable}
\usepackage{listings}
\usepackage{listingsutf8}
\usepackage{framed}
\usepackage{float}
\usepackage{hyperref}
\usepackage{amsmath}

\begin{document}

\begin{titlepage}

\begin{center}
\vspace*{-1in}
\begin{figure}[htb]
\begin{center}
\includegraphics[width=8cm]{logoUZ.png}
\end{center}
\end{figure}

\vspace*{0.3in}

UNIVERSIDAD DE ZARAGOZA \\

\vspace*{0.3in}

\begin{large}
SISTEMAS Y TECNOLOGÍAS WEB\\
\end{large}
\vspace*{0.2in}
\begin{Large}
\textbf{FelinoTweets} \\
\end{Large}
\vspace*{0.3in}
\begin{large}
\end{large}
\vspace*{0.5in}
\rule{80mm}{0.1mm}\\
\vspace*{0.1in}
\begin{large}
Autores: \\
Alejandro Márquez Ferrer (NIP: 566400)\\
Jaime Ruiz-Borau Vizárraga (NIP: 546751)\\
Alejandro Royo Amondarain (NIP: 560285)\\

\end{large}
\end{center}

\end{titlepage}
\tableofcontents

\newpage
\section{Resumen}

	\paragraph{} El trabajo propuesto tiene como objetivo el desarrollo de un sistema que permita gestionar, mediante una interfaz típica de control de acceso de usuarios, una o varias cuentas de \textit{Twitter}.

\section{Propuestas similares - Hootsuite}

	\paragraph{} Como principal aplicación parecida a la propuesta destaca \textit{Hootsuite}. Esta herramienta ofrece un panel de control (\textit{dashboard}) para gestionar las cuentas de diferentes redes sociales (\textit{Twitter}, \textit{Facebook}...).
	
	\paragraph{} Se proporcionan varias versiones de esta aplicación, una gratuita (con funcionalidades limitadas) y otra de pago que ofrece al usuario servicios adicionales como realizar informes de estadísticas sobre sus redes sociales

\section{Arquitectura de alto nivel}
	\paragraph{} A continuación se presenta un diagrama arquitectural de alto nivel con la estructura de la aplicación desarrollada, seguido de una descripción más en detalle de cada una de las partes que la componen:
	\begin{figure}[H]
		\centering
		\includegraphics[width=210px]{diagarq.png}
		\caption{Diagrama arquitectural de alto nivel}
		\label{fig:diagarq}
	\end{figure}
	\newpage
	\subsection{Componentes}
		\begin{itemize}
			\item \textbf{Twitter:} Este componente se encarga de todas las interacciones con Twitter según las solicitudes del Cliente o de otros componentes del servidor. Devuelve y publica tweets y permite realizar retweets o dar "Me gusta".
			\item \textbf{Stats:} Gestiona todas las estadísticas, tanto de usuarios como del administrador del sistema. Se comunica con la base de datos en mLab y con el componente Twitter de FelinoTweets para elaborar las estadísticas.
			\item \textbf{URLs:}
		\end{itemize}
		\paragraph{} La descripción detallada de cada función del API se encuentra en el apartado \textbf{API}.

\section{Modelo de datos (Esquemas Mongoose*)}

\section{API REST}

\section{Analíticas}

\section{Implementación}

\section{Modelo de navegación}

\section{Despliegue del sistema (instrucciones)}

\section{Validación (test, pruebas realizadas)}

\section{Análisis de problemas}

\section{Distribución de tiempos}
	\subsection{Diagrama de Gannt}
	\subsection{Horas de esfuerzo del equipo y por separado en tareas}

\section{Conclusiones}
	\subsection{Conclusiones del proyecto}
	\subsection{Valoración personal del grupo}
	\subsection{Valoración personal de cada miembro}
\section{Anexos}

\begin{figure}[H]
	\centering
	\includegraphics[width=1\linewidth]{logoUZ.png}
	\caption{Caption}
	\label{fig:pf}
\end{figure}

\newpage
\begin{thebibliography}{99} 
\bibitem{paraQueSirve} \textbf{Title} - Consulted in XXXXX aaaa. [\url{link}]

\end{thebibliography}

\end{document}